%============================================================================
%   - Internship report template for ENSAI's students
%
%   - This template can be downloaded from: https://github.com/Lui5ito/EnsaiTemplates
%
%   - Author: Louis Allain
%
%   - License: MIT License
%============================================================================

%============================================================================
%   Write a chapter in this file
%============================================================================


\chapter{My first chapter}
In this chapter you will see examples of how to make a table and show a figure.

\section{A first section}
Here is the first table of the report, \autoref{tab:first_table}. 
\begin{table}[h]
    \centering
    \begin{tabular}{ccc}
    First column & Second column & Third column \\
    1            & 10            & 100          \\
    2            & 20            & 200          \\
    3            & 30            & 300          \\
    4            & 40            & 400          \\
    5            & 50            & 500         
    \end{tabular}
    \caption{First table example, with random numbers.}
    \label{tab:first_table}
\end{table}
You can also make multiple subtables within one table, which are \autoref{subtab:first_subtable} and \autoref{subtab:second_subtable}.
\begin{table}[h]
    \centering
    \begin{subtable}{0.49\linewidth}
        \centering
        \begin{tabular}{|c|c|c|}
        \hline
        First column & Second column & Third column \\ \hline
        1            & 10            & 100          \\ \hline
        2            & 20            & 200          \\ \hline
        3            & 30            & 300          \\ \hline
        4            & 40            & 400          \\ \hline
        5            & 50            & 500          \\ \hline
        \end{tabular}
        \caption{First subtable, with borders.}
        \label{subtab:first_subtable}
    \end{subtable}
    \hfill
    \begin{subtable}{0.49\linewidth}
      \centering
      \begin{tabular}{ccc}
      First column & Second column & Third column \\
      1            & 10            & 100          \\
      2            & 20            & 200          \\
      3            & 30            & 300          \\
      4            & 40            & 400          \\
      5            & 50            & 500         
      \end{tabular}
      \caption{Second subtable, without borders.}
      \label{subtab:second_subtable}
    \end{subtable}
    \caption{An example of a table with two subtables.}
    \label{tab:second_table}
  \end{table}
\section{Another section with subsection}

